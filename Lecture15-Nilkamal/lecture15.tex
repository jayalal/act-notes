\documentclass[11pt]{article}
\usepackage{amsmath,amssymb,amsfonts,amsthm,complexity}
\newcommand{\USAT}{{\sf USAT}~}

%%%%%%%%%%%%%%%%%%%%%%%%%%%%%%%%%%%%%%%%%%%%%%%%%%%%%%
%
% This file should be called  preamble.tex 
% Your main LaTeX file should look like :
%
%	\documentclass[11pt]{article}
%	\usepackage{amsmath,amssymb,amsfonts,amsthm,complexity}
%
%	%%%%%%%%%%%%%%%%%%%%%%%%%%%%%%%%%%%%%%%%%%%%%%%%%%%%%%
%
% This file should be called  preamble.tex 
% Your main LaTeX file should look like :
%
%	\documentclass[11pt]{article}
%	\usepackage{amsmath,amssymb,amsfonts,amsthm,complexity}
%
%	%%%%%%%%%%%%%%%%%%%%%%%%%%%%%%%%%%%%%%%%%%%%%%%%%%%%%%
%
% This file should be called  preamble.tex 
% Your main LaTeX file should look like :
%
%	\documentclass[11pt]{article}
%	\usepackage{amsmath,amssymb,amsfonts,amsthm,complexity}
%
%	\input{preamble.tex}
%	\begin{document}
%
%	\lecture{LECTURE NO.}{TITLE OF SCRIBE}{DATE}{Jayalal Sarma M.N.}{YOUR NAME}
%	\theme{THEME FOCUS}
%	\lectureplan{A BRIEF DESCRIPTION OF THE LECTURE}
%	
%	\section{TOPIC 1}
%	...
%	\section{TOPIC 2}
% 	...
%	\end{document}
% If there is not title, leave it as {}
 

\newtheorem{theorem}{Theorem}
\newtheorem{corollary}[theorem]{Corollary}
\newtheorem{lemma}[theorem]{Lemma}
\newtheorem{observation}[theorem]{Observation}
\newtheorem{proposition}[theorem]{Proposition}
\newtheorem{claim}[theorem]{Claim}
\newtheorem{fact}[theorem]{Fact}
\newtheorem{example}[theorem]{Example}
\newtheorem{assumption}[theorem]{Assumption}

\theoremstyle{definition}
\newtheorem{definition}[theorem]{Definition}

\theoremstyle{remark}
\newtheorem{remark}[theorem]{Remark}

% Setting theorem style back for theorems defined in main file.
\theoremstyle{plain}

%\newenvironment{proof}{\noindent{\bf Proof}\hspace*{1em}}{\qed\bigskip}
\newenvironment{proof-sketch}{\noindent{\bf Sketch of Proof}\hspace*{1em}}{\qed\bigskip}
\newenvironment{proof-idea}{\noindent{\bf Proof Idea}\hspace*{1em}}{\qed\bigskip}
\newenvironment{proof-of-lemma}[1]{\noindent{\bf Proof of Lemma #1}\hspace*{1em}}{\qed\bigskip}
\newenvironment{proof-attempt}{\noindent{\bf Proof Attempt}\hspace*{1em}}{\qed\bigskip}
\newenvironment{proofof}[1]{\noindent{\bf Proof}
of #1:\hspace*{1em}}{\qed\bigskip}

%%%%%%%%%%%%%%%%%%%%%%%%%%%%%%%%%%%%%%%%%%%%%%%%%%%
% Useful Complexity Classes
%%%%%%%%%%%%%%%%%%%%%%%%%%%%%%%%%%%%%%%%%%%%%%%%%%%
\newcommand{\ntime}{\hbox{NTIME}}
\newcommand{\nspace}{\hbox{NSPACE}}
\newcommand{\conspace}{\hbox{co-NSPACE}}
\newcommand{\np}{\hbox{NP}}
\newcommand{\pspace}{\hbox{PSPACE}}
\newcommand{\lspace}{\hbox{L}}
\newcommand{\conp}{\hbox{coNP}}
\newcommand{\exptime}{\hbox{EXPTIME}}
\newcommand{\elem}{\hbox{E}}
\newcommand{\nl}{\hbox{NL}}
\newcommand{\bpp}{\hbox{BPP}}
\newcommand{\nregexp}{\hbox{NREGEXP}}
\newcommand{\tqbf}{\hbox{TQBF}}
\newcommand{\threesat}{\hbox{3SAT}}
\newcommand{\cvp}{\hbox{CVP}}
\newcommand{\stconn}{\hbox{STCONN}}
\newcommand{\ispath}{\hbox{ISPATH}}

%\newcommand{\class}{\hbox{$\mathbb{C}$}} 
%\newcommand{\class}{\hbox{$\mathbf{C}$}} 

\newcommand{\lep}{\leq _{\hbox{P}}}
\newcommand{\lel}{\leq _{\hbox{L}}}
\newcommand{\aspace}[1]{{\rm ASPACE}(#1)}
\newcommand{\atime}[1]{{\rm ATIME}(#1)}
\newcommand{\dtime}[1]{{\rm DTIME}(#1)}
\newcommand{\spa}[1]{{\rm SPACE}(#1)}
\newcommand{\ti}[1]{{\rm TIME}(#1)}
\newcommand{\ap}{{\rm AP}}
\newcommand{\al}{{\rm AL}}


%%%%%%%%%%%%%%%%%%%%%%%%%%%%%%%%%%%%%%%%%%%%%%%%%%%
% Useful Macros
%%%%%%%%%%%%%%%%%%%%%%%%%%%%%%%%%%%%%%%%%%%%%%%%%%%
\renewcommand{\Pr}[1]{\mathify{\mbox{Pr}\left[#1\right]}}
\newcommand{\Exp}[1]{\mathify{\mbox{Exp}\left[#1\right]}}
\newcommand{\bigO}O
\newcommand{\set}[1]{\mathify{\left\{ #1 \right\}}}
\def\half{\frac{1}{2}}

\def\implies{\Rightarrow}
\def\prob#1#2{{\mathop{{\rm Prob}}_{#1}}\left[#2 \right]}
\def\var#1#2{{\mathop{{\rm Var}}_{#1}}[#2]}
\def\expec#1#2{{\mathop{{\rm E}}_{#1}}[#2]}
\def\sizeof#1{\left| #1\right|}
\def\setof#1{\left\{ #1\right\}  }

\newcommand{\F}{{\mathbb{F}}}
\newcommand{\Z}{{\mathbb{Z}}}
%\newcommand{\qed}{\rule{7pt}{7pt}}

% \makeatletter
% \@addtoreset{figure}{section}
% \@addtoreset{table}{section}
% \@addtoreset{equation}{section}
% \makeatother

\newcommand{\FOR}{{\bf for}}
\newcommand{\TO}{{\bf to}}
\newcommand{\DO}{{\bf do}}
\newcommand{\WHILE}{{\bf while}}
\newcommand{\AND}{{\bf and}}
\newcommand{\IF}{{\bf if}}
\newcommand{\THEN}{{\bf then}}
\newcommand{\ELSE}{{\bf else}}
\newcommand{\N}{\mathbb{N}}

% \renewcommand{\thefigure}{\thesection.\arabic{figure}}
% \renewcommand{\thetable}{\thesection.\arabic{table}}
% \renewcommand{\theequation}{\thesection.\arabic{equation}}

% Calligraphic letters
\newcommand{\calA}{{\cal A}}
\newcommand{\calB}{{\cal B}}
\newcommand{\calC}{{\cal C}}
\newcommand{\calD}{{\cal D}}
\newcommand{\calE}{{\cal E}}
\newcommand{\calF}{{\cal F}}
\newcommand{\calG}{{\cal G}}
\newcommand{\calH}{{\cal H}}
\newcommand{\calI}{{\cal I}}
\newcommand{\calJ}{{\cal J}}
\newcommand{\calK}{{\cal K}}
\newcommand{\calL}{{\cal L}}
\newcommand{\calM}{{\cal M}}
\newcommand{\calN}{{\cal N}}
\newcommand{\calO}{{\cal O}}
\newcommand{\calP}{{\cal P}}
\newcommand{\calQ}{{\cal Q}}
\newcommand{\calR}{{\cal R}}
\newcommand{\calS}{{\cal S}}
\newcommand{\calT}{{\cal T}}
\newcommand{\calU}{{\cal U}}
\newcommand{\calV}{{\cal V}}
\newcommand{\calW}{{\cal W}}
\newcommand{\calX}{{\cal X}}
\newcommand{\calY}{{\cal Y}}
\newcommand{\calZ}{{\cal Z}}


% Some macro's from Speilman's course.

\makeatletter
\def\fnum@figure{{\bf Figure \thefigure}}
\def\fnum@table{{\bf Table \thetable}}
\long\def\@mycaption#1[#2]#3{\addcontentsline{\csname
  ext@#1\endcsname}{#1}{\protect\numberline{\csname 
  the#1\endcsname}{\ignorespaces #2}}\par
  \begingroup
    \@parboxrestore
    \small
    \@makecaption{\csname fnum@#1\endcsname}{\ignorespaces #3}\par
  \endgroup}
\def\mycaption{\refstepcounter\@captype \@dblarg{\@mycaption\@captype}}
\makeatother

\newcommand{\figcaption}[1]{\mycaption[]{#1}}
\newcommand{\tabcaption}[1]{\mycaption[]{#1}}


%%%%%%%%%%%%%%%%%%%%%%%%%%%%%%%%%%%%%%%%%%%%%%%%%%%%%%%%%%%%%%%%%%%%%%
% Feel free to ignore the rest of this file.

%%%%%%%%%%%%%%%%%%%%%%%%%%%%%%%
% Margins and page indentations
% DO NOT EDIT
%%%%%%%%%%%%%%%%%%%%%%%%%%%%%%
\textwidth=6in
\oddsidemargin=0.25in
\evensidemargin=0.25in
\topmargin=-0.1in
\footskip=0.8in
\parindent=0.0cm
\parskip=0.3cm
\textheight=8.00in
\setcounter{tocdepth} {3}
\setcounter{secnumdepth} {2}
\sloppy


\newcommand{\handout}[5]{
   \renewcommand{\thepage}{#1-\arabic{page}}
   \noindent
   \begin{center}
   \framebox{
      \vbox{
    \hbox to 5.78in { {\bf CS6840: Advanced Complexity Theory} \hfill #2 }
       \vspace{4mm}
       \hbox to 5.78in { {\Large \hfill #5  \hfill} }
       \vspace{2mm}
       \hbox to 5.78in { {\em #3 \hfill #4} }
      }
   }
   \end{center}
   \vspace*{4mm}
}

\newcommand{\lecture}[5]{\handout{#1}{#3}{Lecturer:~#4}{Scribe: #5}{Lecture~#1~: #2}}

\newcommand{\lectureplan}[1]{{\sc Lecture Plan:}#1\\}
\newcommand{\theme}[1]{{\sc Theme:} #1\\}

%	\begin{document}
%
%	\lecture{LECTURE NO.}{TITLE OF SCRIBE}{DATE}{Jayalal Sarma M.N.}{YOUR NAME}
%	\theme{THEME FOCUS}
%	\lectureplan{A BRIEF DESCRIPTION OF THE LECTURE}
%	
%	\section{TOPIC 1}
%	...
%	\section{TOPIC 2}
% 	...
%	\end{document}
% If there is not title, leave it as {}
 

\newtheorem{theorem}{Theorem}
\newtheorem{corollary}[theorem]{Corollary}
\newtheorem{lemma}[theorem]{Lemma}
\newtheorem{observation}[theorem]{Observation}
\newtheorem{proposition}[theorem]{Proposition}
\newtheorem{claim}[theorem]{Claim}
\newtheorem{fact}[theorem]{Fact}
\newtheorem{example}[theorem]{Example}
\newtheorem{assumption}[theorem]{Assumption}

\theoremstyle{definition}
\newtheorem{definition}[theorem]{Definition}

\theoremstyle{remark}
\newtheorem{remark}[theorem]{Remark}

% Setting theorem style back for theorems defined in main file.
\theoremstyle{plain}

%\newenvironment{proof}{\noindent{\bf Proof}\hspace*{1em}}{\qed\bigskip}
\newenvironment{proof-sketch}{\noindent{\bf Sketch of Proof}\hspace*{1em}}{\qed\bigskip}
\newenvironment{proof-idea}{\noindent{\bf Proof Idea}\hspace*{1em}}{\qed\bigskip}
\newenvironment{proof-of-lemma}[1]{\noindent{\bf Proof of Lemma #1}\hspace*{1em}}{\qed\bigskip}
\newenvironment{proof-attempt}{\noindent{\bf Proof Attempt}\hspace*{1em}}{\qed\bigskip}
\newenvironment{proofof}[1]{\noindent{\bf Proof}
of #1:\hspace*{1em}}{\qed\bigskip}

%%%%%%%%%%%%%%%%%%%%%%%%%%%%%%%%%%%%%%%%%%%%%%%%%%%
% Useful Complexity Classes
%%%%%%%%%%%%%%%%%%%%%%%%%%%%%%%%%%%%%%%%%%%%%%%%%%%
\newcommand{\ntime}{\hbox{NTIME}}
\newcommand{\nspace}{\hbox{NSPACE}}
\newcommand{\conspace}{\hbox{co-NSPACE}}
\newcommand{\np}{\hbox{NP}}
\newcommand{\pspace}{\hbox{PSPACE}}
\newcommand{\lspace}{\hbox{L}}
\newcommand{\conp}{\hbox{coNP}}
\newcommand{\exptime}{\hbox{EXPTIME}}
\newcommand{\elem}{\hbox{E}}
\newcommand{\nl}{\hbox{NL}}
\newcommand{\bpp}{\hbox{BPP}}
\newcommand{\nregexp}{\hbox{NREGEXP}}
\newcommand{\tqbf}{\hbox{TQBF}}
\newcommand{\threesat}{\hbox{3SAT}}
\newcommand{\cvp}{\hbox{CVP}}
\newcommand{\stconn}{\hbox{STCONN}}
\newcommand{\ispath}{\hbox{ISPATH}}

%\newcommand{\class}{\hbox{$\mathbb{C}$}} 
%\newcommand{\class}{\hbox{$\mathbf{C}$}} 

\newcommand{\lep}{\leq _{\hbox{P}}}
\newcommand{\lel}{\leq _{\hbox{L}}}
\newcommand{\aspace}[1]{{\rm ASPACE}(#1)}
\newcommand{\atime}[1]{{\rm ATIME}(#1)}
\newcommand{\dtime}[1]{{\rm DTIME}(#1)}
\newcommand{\spa}[1]{{\rm SPACE}(#1)}
\newcommand{\ti}[1]{{\rm TIME}(#1)}
\newcommand{\ap}{{\rm AP}}
\newcommand{\al}{{\rm AL}}


%%%%%%%%%%%%%%%%%%%%%%%%%%%%%%%%%%%%%%%%%%%%%%%%%%%
% Useful Macros
%%%%%%%%%%%%%%%%%%%%%%%%%%%%%%%%%%%%%%%%%%%%%%%%%%%
\renewcommand{\Pr}[1]{\mathify{\mbox{Pr}\left[#1\right]}}
\newcommand{\Exp}[1]{\mathify{\mbox{Exp}\left[#1\right]}}
\newcommand{\bigO}O
\newcommand{\set}[1]{\mathify{\left\{ #1 \right\}}}
\def\half{\frac{1}{2}}

\def\implies{\Rightarrow}
\def\prob#1#2{{\mathop{{\rm Prob}}_{#1}}\left[#2 \right]}
\def\var#1#2{{\mathop{{\rm Var}}_{#1}}[#2]}
\def\expec#1#2{{\mathop{{\rm E}}_{#1}}[#2]}
\def\sizeof#1{\left| #1\right|}
\def\setof#1{\left\{ #1\right\}  }

\newcommand{\F}{{\mathbb{F}}}
\newcommand{\Z}{{\mathbb{Z}}}
%\newcommand{\qed}{\rule{7pt}{7pt}}

% \makeatletter
% \@addtoreset{figure}{section}
% \@addtoreset{table}{section}
% \@addtoreset{equation}{section}
% \makeatother

\newcommand{\FOR}{{\bf for}}
\newcommand{\TO}{{\bf to}}
\newcommand{\DO}{{\bf do}}
\newcommand{\WHILE}{{\bf while}}
\newcommand{\AND}{{\bf and}}
\newcommand{\IF}{{\bf if}}
\newcommand{\THEN}{{\bf then}}
\newcommand{\ELSE}{{\bf else}}
\newcommand{\N}{\mathbb{N}}

% \renewcommand{\thefigure}{\thesection.\arabic{figure}}
% \renewcommand{\thetable}{\thesection.\arabic{table}}
% \renewcommand{\theequation}{\thesection.\arabic{equation}}

% Calligraphic letters
\newcommand{\calA}{{\cal A}}
\newcommand{\calB}{{\cal B}}
\newcommand{\calC}{{\cal C}}
\newcommand{\calD}{{\cal D}}
\newcommand{\calE}{{\cal E}}
\newcommand{\calF}{{\cal F}}
\newcommand{\calG}{{\cal G}}
\newcommand{\calH}{{\cal H}}
\newcommand{\calI}{{\cal I}}
\newcommand{\calJ}{{\cal J}}
\newcommand{\calK}{{\cal K}}
\newcommand{\calL}{{\cal L}}
\newcommand{\calM}{{\cal M}}
\newcommand{\calN}{{\cal N}}
\newcommand{\calO}{{\cal O}}
\newcommand{\calP}{{\cal P}}
\newcommand{\calQ}{{\cal Q}}
\newcommand{\calR}{{\cal R}}
\newcommand{\calS}{{\cal S}}
\newcommand{\calT}{{\cal T}}
\newcommand{\calU}{{\cal U}}
\newcommand{\calV}{{\cal V}}
\newcommand{\calW}{{\cal W}}
\newcommand{\calX}{{\cal X}}
\newcommand{\calY}{{\cal Y}}
\newcommand{\calZ}{{\cal Z}}


% Some macro's from Speilman's course.

\makeatletter
\def\fnum@figure{{\bf Figure \thefigure}}
\def\fnum@table{{\bf Table \thetable}}
\long\def\@mycaption#1[#2]#3{\addcontentsline{\csname
  ext@#1\endcsname}{#1}{\protect\numberline{\csname 
  the#1\endcsname}{\ignorespaces #2}}\par
  \begingroup
    \@parboxrestore
    \small
    \@makecaption{\csname fnum@#1\endcsname}{\ignorespaces #3}\par
  \endgroup}
\def\mycaption{\refstepcounter\@captype \@dblarg{\@mycaption\@captype}}
\makeatother

\newcommand{\figcaption}[1]{\mycaption[]{#1}}
\newcommand{\tabcaption}[1]{\mycaption[]{#1}}


%%%%%%%%%%%%%%%%%%%%%%%%%%%%%%%%%%%%%%%%%%%%%%%%%%%%%%%%%%%%%%%%%%%%%%
% Feel free to ignore the rest of this file.

%%%%%%%%%%%%%%%%%%%%%%%%%%%%%%%
% Margins and page indentations
% DO NOT EDIT
%%%%%%%%%%%%%%%%%%%%%%%%%%%%%%
\textwidth=6in
\oddsidemargin=0.25in
\evensidemargin=0.25in
\topmargin=-0.1in
\footskip=0.8in
\parindent=0.0cm
\parskip=0.3cm
\textheight=8.00in
\setcounter{tocdepth} {3}
\setcounter{secnumdepth} {2}
\sloppy


\newcommand{\handout}[5]{
   \renewcommand{\thepage}{#1-\arabic{page}}
   \noindent
   \begin{center}
   \framebox{
      \vbox{
    \hbox to 5.78in { {\bf CS6840: Advanced Complexity Theory} \hfill #2 }
       \vspace{4mm}
       \hbox to 5.78in { {\Large \hfill #5  \hfill} }
       \vspace{2mm}
       \hbox to 5.78in { {\em #3 \hfill #4} }
      }
   }
   \end{center}
   \vspace*{4mm}
}

\newcommand{\lecture}[5]{\handout{#1}{#3}{Lecturer:~#4}{Scribe: #5}{Lecture~#1~: #2}}

\newcommand{\lectureplan}[1]{{\sc Lecture Plan:}#1\\}
\newcommand{\theme}[1]{{\sc Theme:} #1\\}

%	\begin{document}
%
%	\lecture{LECTURE NO.}{TITLE OF SCRIBE}{DATE}{Jayalal Sarma M.N.}{YOUR NAME}
%	\theme{THEME FOCUS}
%	\lectureplan{A BRIEF DESCRIPTION OF THE LECTURE}
%	
%	\section{TOPIC 1}
%	...
%	\section{TOPIC 2}
% 	...
%	\end{document}
% If there is not title, leave it as {}
 

\newtheorem{theorem}{Theorem}
\newtheorem{corollary}[theorem]{Corollary}
\newtheorem{lemma}[theorem]{Lemma}
\newtheorem{observation}[theorem]{Observation}
\newtheorem{proposition}[theorem]{Proposition}
\newtheorem{claim}[theorem]{Claim}
\newtheorem{fact}[theorem]{Fact}
\newtheorem{example}[theorem]{Example}
\newtheorem{assumption}[theorem]{Assumption}

\theoremstyle{definition}
\newtheorem{definition}[theorem]{Definition}

\theoremstyle{remark}
\newtheorem{remark}[theorem]{Remark}

% Setting theorem style back for theorems defined in main file.
\theoremstyle{plain}

%\newenvironment{proof}{\noindent{\bf Proof}\hspace*{1em}}{\qed\bigskip}
\newenvironment{proof-sketch}{\noindent{\bf Sketch of Proof}\hspace*{1em}}{\qed\bigskip}
\newenvironment{proof-idea}{\noindent{\bf Proof Idea}\hspace*{1em}}{\qed\bigskip}
\newenvironment{proof-of-lemma}[1]{\noindent{\bf Proof of Lemma #1}\hspace*{1em}}{\qed\bigskip}
\newenvironment{proof-attempt}{\noindent{\bf Proof Attempt}\hspace*{1em}}{\qed\bigskip}
\newenvironment{proofof}[1]{\noindent{\bf Proof}
of #1:\hspace*{1em}}{\qed\bigskip}

%%%%%%%%%%%%%%%%%%%%%%%%%%%%%%%%%%%%%%%%%%%%%%%%%%%
% Useful Complexity Classes
%%%%%%%%%%%%%%%%%%%%%%%%%%%%%%%%%%%%%%%%%%%%%%%%%%%
\newcommand{\ntime}{\hbox{NTIME}}
\newcommand{\nspace}{\hbox{NSPACE}}
\newcommand{\conspace}{\hbox{co-NSPACE}}
\newcommand{\np}{\hbox{NP}}
\newcommand{\pspace}{\hbox{PSPACE}}
\newcommand{\lspace}{\hbox{L}}
\newcommand{\conp}{\hbox{coNP}}
\newcommand{\exptime}{\hbox{EXPTIME}}
\newcommand{\elem}{\hbox{E}}
\newcommand{\nl}{\hbox{NL}}
\newcommand{\bpp}{\hbox{BPP}}
\newcommand{\nregexp}{\hbox{NREGEXP}}
\newcommand{\tqbf}{\hbox{TQBF}}
\newcommand{\threesat}{\hbox{3SAT}}
\newcommand{\cvp}{\hbox{CVP}}
\newcommand{\stconn}{\hbox{STCONN}}
\newcommand{\ispath}{\hbox{ISPATH}}

%\newcommand{\class}{\hbox{$\mathbb{C}$}} 
%\newcommand{\class}{\hbox{$\mathbf{C}$}} 

\newcommand{\lep}{\leq _{\hbox{P}}}
\newcommand{\lel}{\leq _{\hbox{L}}}
\newcommand{\aspace}[1]{{\rm ASPACE}(#1)}
\newcommand{\atime}[1]{{\rm ATIME}(#1)}
\newcommand{\dtime}[1]{{\rm DTIME}(#1)}
\newcommand{\spa}[1]{{\rm SPACE}(#1)}
\newcommand{\ti}[1]{{\rm TIME}(#1)}
\newcommand{\ap}{{\rm AP}}
\newcommand{\al}{{\rm AL}}


%%%%%%%%%%%%%%%%%%%%%%%%%%%%%%%%%%%%%%%%%%%%%%%%%%%
% Useful Macros
%%%%%%%%%%%%%%%%%%%%%%%%%%%%%%%%%%%%%%%%%%%%%%%%%%%
\renewcommand{\Pr}[1]{\mathify{\mbox{Pr}\left[#1\right]}}
\newcommand{\Exp}[1]{\mathify{\mbox{Exp}\left[#1\right]}}
\newcommand{\bigO}O
\newcommand{\set}[1]{\mathify{\left\{ #1 \right\}}}
\def\half{\frac{1}{2}}

\def\implies{\Rightarrow}
\def\prob#1#2{{\mathop{{\rm Prob}}_{#1}}\left[#2 \right]}
\def\var#1#2{{\mathop{{\rm Var}}_{#1}}[#2]}
\def\expec#1#2{{\mathop{{\rm E}}_{#1}}[#2]}
\def\sizeof#1{\left| #1\right|}
\def\setof#1{\left\{ #1\right\}  }

\newcommand{\F}{{\mathbb{F}}}
\newcommand{\Z}{{\mathbb{Z}}}
%\newcommand{\qed}{\rule{7pt}{7pt}}

% \makeatletter
% \@addtoreset{figure}{section}
% \@addtoreset{table}{section}
% \@addtoreset{equation}{section}
% \makeatother

\newcommand{\FOR}{{\bf for}}
\newcommand{\TO}{{\bf to}}
\newcommand{\DO}{{\bf do}}
\newcommand{\WHILE}{{\bf while}}
\newcommand{\AND}{{\bf and}}
\newcommand{\IF}{{\bf if}}
\newcommand{\THEN}{{\bf then}}
\newcommand{\ELSE}{{\bf else}}
\newcommand{\N}{\mathbb{N}}

% \renewcommand{\thefigure}{\thesection.\arabic{figure}}
% \renewcommand{\thetable}{\thesection.\arabic{table}}
% \renewcommand{\theequation}{\thesection.\arabic{equation}}

% Calligraphic letters
\newcommand{\calA}{{\cal A}}
\newcommand{\calB}{{\cal B}}
\newcommand{\calC}{{\cal C}}
\newcommand{\calD}{{\cal D}}
\newcommand{\calE}{{\cal E}}
\newcommand{\calF}{{\cal F}}
\newcommand{\calG}{{\cal G}}
\newcommand{\calH}{{\cal H}}
\newcommand{\calI}{{\cal I}}
\newcommand{\calJ}{{\cal J}}
\newcommand{\calK}{{\cal K}}
\newcommand{\calL}{{\cal L}}
\newcommand{\calM}{{\cal M}}
\newcommand{\calN}{{\cal N}}
\newcommand{\calO}{{\cal O}}
\newcommand{\calP}{{\cal P}}
\newcommand{\calQ}{{\cal Q}}
\newcommand{\calR}{{\cal R}}
\newcommand{\calS}{{\cal S}}
\newcommand{\calT}{{\cal T}}
\newcommand{\calU}{{\cal U}}
\newcommand{\calV}{{\cal V}}
\newcommand{\calW}{{\cal W}}
\newcommand{\calX}{{\cal X}}
\newcommand{\calY}{{\cal Y}}
\newcommand{\calZ}{{\cal Z}}


% Some macro's from Speilman's course.

\makeatletter
\def\fnum@figure{{\bf Figure \thefigure}}
\def\fnum@table{{\bf Table \thetable}}
\long\def\@mycaption#1[#2]#3{\addcontentsline{\csname
  ext@#1\endcsname}{#1}{\protect\numberline{\csname 
  the#1\endcsname}{\ignorespaces #2}}\par
  \begingroup
    \@parboxrestore
    \small
    \@makecaption{\csname fnum@#1\endcsname}{\ignorespaces #3}\par
  \endgroup}
\def\mycaption{\refstepcounter\@captype \@dblarg{\@mycaption\@captype}}
\makeatother

\newcommand{\figcaption}[1]{\mycaption[]{#1}}
\newcommand{\tabcaption}[1]{\mycaption[]{#1}}


%%%%%%%%%%%%%%%%%%%%%%%%%%%%%%%%%%%%%%%%%%%%%%%%%%%%%%%%%%%%%%%%%%%%%%
% Feel free to ignore the rest of this file.

%%%%%%%%%%%%%%%%%%%%%%%%%%%%%%%
% Margins and page indentations
% DO NOT EDIT
%%%%%%%%%%%%%%%%%%%%%%%%%%%%%%
\textwidth=6in
\oddsidemargin=0.25in
\evensidemargin=0.25in
\topmargin=-0.1in
\footskip=0.8in
\parindent=0.0cm
\parskip=0.3cm
\textheight=8.00in
\setcounter{tocdepth} {3}
\setcounter{secnumdepth} {2}
\sloppy


\newcommand{\handout}[5]{
   \renewcommand{\thepage}{#1-\arabic{page}}
   \noindent
   \begin{center}
   \framebox{
      \vbox{
    \hbox to 5.78in { {\bf CS6840: Advanced Complexity Theory} \hfill #2 }
       \vspace{4mm}
       \hbox to 5.78in { {\Large \hfill #5  \hfill} }
       \vspace{2mm}
       \hbox to 5.78in { {\em #3 \hfill #4} }
      }
   }
   \end{center}
   \vspace*{4mm}
}

\newcommand{\lecture}[5]{\handout{#1}{#3}{Lecturer:~#4}{Scribe: #5}{Lecture~#1~: #2}}

\newcommand{\lectureplan}[1]{{\sc Lecture Plan:}#1\\}
\newcommand{\theme}[1]{{\sc Theme:} #1\\}

\begin{document}
\lecture{15}{ Introduction to Toda's Theorem}{Jan 31, 2012}{ Jayalal Sarma M.N.}{Nilkamal Adak}
\theme{Between $\P$ and $\PSPACE$.}
\lectureplan{Introduce Toda's Theorem, give some relevant definitions 
and construct a proof strategy for Toda's Theorem}

There are two different way to generalize hard problems like optimization 
problems.
\begin{enumerate}
 \item Non-determinism or quantification
 \item Counting
\end{enumerate}

It was an important open question in 1980's about relative power of those 
two approach, alternation and counting. There are various 
classes are defined independently in those two different approach. But how 
are those class comparable? In 1991, Toda came up with the following relation 
among the two different world.

\begin{theorem}
 $PH \subseteq \P^{\#\P}$
\end{theorem}
This theorem take the whole quantification world to counting world. That is 
we can solve any problem in PH in polynomial with an oracle access to 
a $\#\P$-Complete problem. Before going to the proof of Toda's theorem lets 
define the following.

\begin{definition}
$\exists.$ Operator.

Let us consider the definition of $\NP$.
\begin{center}
$L \in \NP$ if $ \exists B \in \P$ such that 
\[ x \in L \Longleftrightarrow \exists y \text{ such that } (x,y) \in B \]
\end{center}
\end{definition}
So we can think $\exists$ as an operator and let us define it as follows

Let $C$ be any class of languages. Define, 
\begin{center}
$ L \in \exists.C$  if $\exists B \in C$  such that
$x \in L \Longleftrightarrow \exists y$ such that $(x,y) \in B$
\end{center}
So $\NP = \exists .\P$

\begin{definition}
BP. Operator \\
Let us consider definition of the class $\BPP$ 

$L \in \BPP$ if $\exists B \in \P$ such that 
\begin{align*}
x \in L & \Longrightarrow {\sf Pr_y}[(x,y) \in B ] \ge \frac{3}{4} \\
x \notin L & \Longrightarrow {\sf Pr_y}[(x,y) \in B] \le \frac{1}{4}
\end{align*}
So similarly we can define BP. operator as 


\begin{center}
Let $C$ be a class of languages. 
$L \in BP.C$ if $ \exists B \in$ C such that
\begin{align*}
x \in L & \Longrightarrow {\sf Pr}[(x,y) \in B]  \ge \frac{3}{4}\\
x \notin L & \Longrightarrow {\sf Pr}[(x,y) \in B] \le \frac{1}{4}
\end{align*}
\end{center}
So $\BPP = BP.\P$
\end{definition}

Now lets define the class $\oplus \P$:
\begin{definition}
L $\in \oplus \P$ if there is a branching machine $M$ such that
\[ x \in L \Longrightarrow \#Acc_M(x) \text{ is odd} \]
\end{definition}

$\oplus\P$ can be considered as the class of decision problems corresponding 
to the least significant bit of a $\#\P$-problem. Now the natural 
question is ``Is $\NP \subseteq \oplus\P$?''.

We will see that $NP$ problems are reduced to $\oplus\P$ in some extend.
Lets now define Randomize Reduction as follows
\begin{definition}
$A \le B$ via a randomize reduction function $\sigma$ if 
\begin{enumerate}
\item  $\sigma$ is computable by choosing a random string 
$y \in \{0,1\}^{p(n)}$.
\item $x \in L \iff \sigma_y(x) \in B$ with high probability.
\end{enumerate}
\end{definition}

We will see that $\NP$ problems are reduced to some $\oplus\P$ problem 
via randomize reduction. More specifically $\NP \subseteq BP.(\oplus \P)$.
Now we will develop the proof strategy for Toda's theorem. 
We will proof the theorem in the following steps $\ldots$ 
\[
\NP \subseteq BP.(\oplus \P) \subseteq \BPP^{\oplus \P} \subseteq 
\P^{\#\P}=\P^{\PP} \]
Then by induction on the number of quantifiers we will show that the 
entire $PH \subseteq \P^{\#\P}$.

So our first step is to show $\NP \subseteq BP.(\oplus \P)$.

To prove this claim let us define the following languages \ldots.
\begin{definition} 
\[ \oplus \SAT = \{ \phi : \# \text{ of satisfying assignment of }
\phi \text{ is odd} \}\]
\end{definition}

Let us define the following promise problem \USAT as,
\begin{definition}
For all boolean formulas $\phi$,
\begin{align*}
\phi & \in \USAT \Longrightarrow \#SAT(\phi)=1\\
\phi & \notin \USAT \Longrightarrow \#SAT(\phi)=0
\end{align*}
\end{definition}

This is a promise problem as we will consider only those $\phi$ which 
falls into above two cases. Now we introduce a very important result called 
\emph{Valiant-Vazirani lemma} proved by Valiant and Vazirani in 1986.

\begin{lemma}(Valiant-Vazirani Lemma) \\
There exists a randomize reduction $\sigma_y$ such that
\begin{align*}
\phi & \in \SAT \iff Pr_y[\sigma_y(\phi) \in \USAT] \ge \frac{3}{4}\\
\phi & \notin \SAT \iff Pr_y[\sigma_y(\phi) \in \USAT] \le \frac{1}{4}
\end{align*}
\end{lemma}
We will prove this lemma in the next lecture and using this lemma 
we will prove our first step $\NP \subseteq BP.(\oplus\P)$.
\end{document}
